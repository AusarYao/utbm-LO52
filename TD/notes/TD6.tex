\chapter{Traitement des interférences}
\subsubsection{Question 1}
\begin{align*}
P_{bt} &= kBT\\
       &= 1.3806 . 10^{-23} \times 22 . 10^6 \times 298\\
       &= 9.05 \times 10^{-14} W\\
       &\Rightarrow 10.log(9.05.10^{-11}) = -100.04 dBm
\end{align*}
\subsubsection{Question 2}
Il faut 5 canaux d'écart.

\subsubsection{Question 3}
On peut utiliser jusqu'à 3 AP sans interférences, par exemple les 1, 8 et 13.

\subsubsection{Question 5}
Mick est sur l'AP 4.

\subsubsection{Question 6}
\begin{center}
\begin{tabular}{|c|c|c|}
\hline
AP & Puissance en dBm & Puissance en W\\
\hline
1  &       -70        &   $10^{-10}$\\
\hline
2  &       -84        &   $3.98.10^{-12}$\\
\hline
3  &       -63        &   $5.01.10^{-10}$\\
\hline
4  &       -60        &   $10^{-9}$\\
\hline
5  &       -68        &   $1.58.10^{-10}$\\
\hline
\end{tabular}
\end{center}

\subsubsection{Question 7}
\begin{align*}
SINR &= \frac{10^{-9}}{0.73 \times 3.98.10^{-12} + 0.05 \times 5.01 .
  10^{-10} + 0.01 \times 1.58.10^{-10} + 9.05.10^{-14}}\\
    &= 33.75\\
\end{align*}
Le SINR n'a pas d'unité.\\

Oui.

\subsubsection{Question 8}
\begin{align*}
SINR &= \frac{10^{-9}}{0.05 \times 5.01 . 10^{-10}} + 9.05.10^{-14}\\
    &= 39.77\\
    &\Rightarrow 16 dB
\end{align*}

\subsubsection{Question 9}
Nous aurons un débit nominal utilisé sera 24Mbps.
