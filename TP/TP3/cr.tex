\documentclass[a4paper]{article}

\usepackage[utf8]{inputenc}
\usepackage[francais]{babel}
\usepackage{listings}
\usepackage{color} % on en a besoin pour utiliser les couleurs
\definecolor{grey}{rgb}{0.95,0.95,0.95}

\usepackage[top=2.5cm, bottom=2cm, left=2.5cm, right=2cm]{geometry}

\lstset{tabsize=2, frame=single, basicstyle=\footnotesize, xleftmargin=12mm,
    backgroundcolor=\color{grey}, showtabs=false, showspaces=false,
    showstringspaces=false, numbers=left, numberstyle=\footnotesize,
    framexleftmargin=13mm, language=Java}

\title{TP3 : J2ME\\LO52}
\author{Maxime Ripard}
\date{Printemps 2010}

\begin{document}
\large
\maketitle
\newpage

\section{Objectif du TP}

Il s'agit ici de développer des applications pour téléphone mobile en J2ME.
Le WTK permet de créer de nouveaux projets, de les compiler et de les
exécuter grâce à l'application \og ktoolbar \fg, mais pas d'éditer le code
source correspondant. Il est muni d’un émulateur permettant le test de ces
applications.

\section{Utilisation de WTK}

L'objectif de cette partie est de prendre en main le toolkit, en se faisant
à la navigation, la saisie de texte et la communication par Bluetooth.

\section{Création d'une MiniMIDLet}

La première étape consiste ici à nous faire réaliser un \og Hello World! \fg
. Il nous faut ensuite ajouter un bouton permettant d'afficher ce message
lors de l'appui. Il nous faut pour cela utiliser une instance de la
classe \verb?Command? en utilisant l'identifieur \verb?Command.ITEM?.\\

Il nous fallait ensuite ajouter un bouton de retour à l'état initial du
programme. Nous avons donc ici aussi utilisé la classe \verb?Command?.\\

Voici le programme final:
\lstinputlisting{Tiny/src/TinyMIDlet.java}

\section{Création d'une Application Client/Serveur}
Ce qui nous est demandé ici est mettre en place une connexion entre deux
appareils Bluetooth : un serveur qui met à disposition un service et un
client qui s'y connecte. Les deux spécifient une chaîne hexadécimale pour
s'identifier.
\subsection{Création du serveur}
Ici nous voulons créer un serveur affichant à l'écran chaque information
reçue. Ce serveur héritera de \verb?Runnable?, ce qui signifie qu'il sera
lancé dans un thread séparé, permettant d'être exécuté en tâche de fond par
le système.
\lstinputlisting{BTServer/src/BTServer.java}

\subsection{Création du client}
Le client doit également hériter de \verb?Runnable? et utiliser le Bluetooth.

\lstinputlisting{BTClient/src/BTClient.java}

\subsection{Création d'une application Client/Serveur}
Pour finir, nous avons fait fusionner les deux applications précédentes afin
qu'elles puissent jouer à la fois les rôles de client et de serveur. Pour
cela, l'application tente de se connecter à un serveur. Nous sommes donc en
mode client. Si toutefois cette connexion échoue, nous partons du principe
qu'il n'y a pas de serveur. L'application passe donc en mode serveur,
servant les requêtes des futurs clients.

\lstinputlisting{BTCliServ/src/BTCliServ.java}


\section{Conclusion}
Ce TP aura été l'occasion de prendre en main J2ME, une plateforme largement
diffusée sur les systèmes mobiles. On peut s'apercevoir que la syntaxe et
les paradigmes utilisés restent fidèles à Java, la transition est donc
facilitée d'une plateforme plus traditionnelle. L'API Bluetooth permet de
connecter facilement 2 appareils entre eux afin d'échanger des données.\\

Le J2ME Wireless Toolkit permet également de tester et valider facilement
nos applications, et ce sur plusieurs mobiles simultanément, ce qui permet
donc de pouvoir développer et déployer rapidement des applications à
destination de périphériques mobiles sans avoir à disposition le matériel
nécessaire.\\

Le seul regret que nous pourrions avoir est de n'avoir finalement disposé
que de 3 heures pour se familiariser avec cette plateforme.
\end{document}
