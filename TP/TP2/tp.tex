\documentclass[a4paper]{report}

\usepackage[utf8]{inputenc}
\usepackage[francais]{babel}
\usepackage{listings}
\usepackage{phdthesis}
\usepackage{palatino}
\usepackage{bbm}
\usepackage[top=2.5cm, bottom=2cm, left=2.5cm, right=2cm]{geometry}

\usepackage{color} % on en a besoin pour utiliser les couleurs
\definecolor{grey}{rgb}{0.95,0.95,0.95}

\lstset{tabsize=2, frame=single, basicstyle=\footnotesize, xleftmargin=12mm,
    backgroundcolor=\color{grey}, showtabs=false, showspaces=false,
    showstringspaces=false, numbers=left, numberstyle=\footnotesize,
    framexleftmargin=13mm, language=C}

\title{TP2 : Robots Mindstorms\\LO52}
\author{Hyacinthe Cartiaux\\Maxime Ripard}
\date{Printemps 2010}
\begin{document}
\large
\maketitle
\tableofcontents

\chapter{Robot Explorateur}
Nous devions ici réaliser un programme permettant de se déplacer en évitant
les obstacles. Nous avons pour celà choisi d'utiliser deux moteurs pour le
déplacement et un capteur de contact sur le \og nez \fg du robot, afin de
détecter les obstacles.\\

Notre programme est ici très simple, le robot se contentant d'avancer en
ligne droite jusqu'au premier obstacle, où il effectue une marche arrière
puis un virage comme seule man\oe{}uvre d'évitement, avant de repartir en
marche avant.\\

\lstinputlisting{evite_obstacle.nxc}

\chapter{Suivi de ligne}

Il s'agit ici aussi de réaliser un programme assez simple. Le but était de
suivre une ligne noire tracée au sol. Pour cela, nous avons utilisé là
encore les deux moteurs, ainsi que deux capteurs de luminosité situés à
l'avant de robot et positionnés de part et d'autre de la ligne.\\

Nous partons du principe que le robot est positionné sur la ligne. Nos deux
capteurs sont eux de part et d'autre de la ligne, ils renvoient donc une
valeur correspondante à du blanc. Si toutefois cette valeur venait à changer
pour du noir, cela veut donc dire que nous nous écartons de la ligne. Il
nous faut alors effectuer un léger virage pour nous repositionner sur la
ligne.\\

\lstinputlisting{suis_ligne.nxc}

\chapter{Capteur de proximité}

L'objectif de cette partie était de remplir les deux objectifs précédents, à
savoir suivre une ligne, tout en évitant un obstacle sur celle-ci. Pour
cela, nous avons utilisé un sonar, pour détecter les obstacles, et deux
capteurs de luminosité pour suivre la ligne.\\

Le sonar nous sert donc à détecter la présence d'un obstacle proche. Lors
de la détection, une man\oe{}uvre d'évitement est réalisée, consistant en
un virage, une marche avant puis un second virage. Après ce second virage,
les moteurs sont remis en marche normale, avec une direction les amenant à
croiser de nouveau la ligne. Lors de ce croisement, nous repassons en mode
suivi de ligne, qui nous permet de reprendre notre chemin le long de celle-
ci.\\

\lstinputlisting{ligne_et_evite.nxc}

\chapter{Conclusion}

Ce TP aura été très intéressant et constitue une bonne prise en main pour
aborder le projet. En effet, nous avons pu nous familiariser à travers ces
quelques exercices à la fois au langage \verb?NXC?, utilisé pour la
programmation de la brique, et à l'utilisation des différents capteurs.\\

Nous avons également pu appréhender quelques difficultés liées à
l'environnement qu'il nous faudra gérer au mieux pour le projet, telles que
la motricité et l'adhérence limitées des roues, l'asservissement peu précis
ou encore la variation de la valeur retournée par les capteurs de luminosité
en fonction de la lumière.

\end{document}
